%%%%%%%%%%%%%%%%%%%%%%%%%%%%%%%%%%%%%%%%%
% Arsclassica Article
% LaTeX Template
% Version 1.1 (10/6/14)
%
% This template has been downloaded from:
% http://www.LaTeXTemplates.com
%
% Original author:
% Lorenzo Pantieri (http://www.lorenzopantieri.net) with extensive modifications by:
% Vel (vel@latextemplates.com)
%
% License:
% CC BY-NC-SA 3.0 (http://creativecommons.org/licenses/by-nc-sa/3.0/)
%
%%%%%%%%%%%%%%%%%%%%%%%%%%%%%%%%%%%%%%%%%

%----------------------------------------------------------------------------------------
%	PACKAGES AND OTHER DOCUMENT CONFIGURATIONS
%----------------------------------------------------------------------------------------

\documentclass[
10pt, % Main document font size
a4paper, % Paper type, use 'letterpaper' for US Letter paper
oneside, % One page layout (no page indentation)
%twoside, % Two page layout (page indentation for binding and different headers)
headinclude,footinclude, % Extra spacing for the header and footer
BCOR5mm, % Binding correction
]{scrartcl}

\input{structure.tex} % Include the structure.tex file which specified the document structure and layout

\hyphenation{Fortran hy-phen-ation} % Specify custom hyphenation points in words with dashes where you would like hyphenation to occur, or alternatively, don't put any dashes in a word to stop hyphenation altogether

%----------------------------------------------------------------------------------------
%	TITLE AND AUTHOR(S)
%----------------------------------------------------------------------------------------

\title{\normalfont\spacedallcaps{Whitehouse.gov usability report}} % The article title

\author{\spacedlowsmallcaps{Enrico Rotundo*, 1008052}} % The article author(s) - author affiliations need to be specified in the AUTHOR AFFILIATIONS block

\date{August, 2014} % An optional date to appear under the author(s)

%----------------------------------------------------------------------------------------

\begin{document}

%----------------------------------------------------------------------------------------
%	HEADERS
%----------------------------------------------------------------------------------------

\renewcommand{\sectionmark}[1]{\markright{\spacedlowsmallcaps{#1}}} % The header for all pages (oneside) or for even pages (twoside)
%\renewcommand{\subsectionmark}[1]{\markright{\thesubsection~#1}} % Uncomment when using the twoside option - this modifies the header on odd pages
\lehead{\mbox{\llap{\small\thepage\kern1em\color{halfgray} \vline}\color{halfgray}\hspace{0.5em}\rightmark\hfil}} % The header style

\pagestyle{scrheadings} % Enable the headers specified in this block

\sloppy % solve the problem with overfull boxes (e.g. \href over the hbox)

%----------------------------
% CUSTOM COMMAND
%----------------------------

\newcommand{\thesite}{\href{http://www.whitehouse.gov/}{whitehouse.gov}}

\newcolumntype{C}[1]{>{\Centering}m{#1}} % http://tex.stackexchange.com/questions/15282/tabular-title-above-and-caption-below
\renewcommand\tabularxcolumn[1]{C{#1}}

%----------------------------------------------------------------------------------------
%	TABLE OF CONTENTS & LISTS OF FIGURES AND TABLES
%----------------------------------------------------------------------------------------

\maketitle % Print the title/author/date block

\setcounter{tocdepth}{2} % Set the depth of the table of contents to show sections and subsections only

\tableofcontents % Print the table of contents

\listoffigures % Print the list of figures

% \listoftables % Print the list of tables

%----------------------------------------------------------------------------------------
%	ABSTRACT
%----------------------------------------------------------------------------------------

%\section*{Abstract} % This section will not appear in the table of contents due to the star (\section*)

%\lipsum[1] % Dummy text

%----------------------------------------------------------------------------------------
%	AUTHOR AFFILIATIONS
%----------------------------------------------------------------------------------------

{\let\thefootnote\relax\footnotetext{* \textit{Computer Science BSc student, AY 2013-2014, Department of Mathematics, Univerisity of Padua, Italy}}}

%----------------------------------------------------------------------------------------

\newpage % Start the article content on the second page, remove this if you have a longer abstract that goes onto the second page

%----------------------------------------------------------------------------------------
%	INTRODUCTION
%----------------------------------------------------------------------------------------

\section{Introduction}
	This documents is a usability report of the website \thesite{}, following referred as ``the site'' or ``the webiste''. The website analized represent the US government, it's focused on the person of the President and deliver high grained informations about him and Administration activities.  
 
%----------------------------------------------------------------------------------------
%	NAME AND DOMAIN
%----------------------------------------------------------------------------------------

\section{Name and domain}

The website name is the same of the institutions that represent, its meaning is unambiguous all around the world, so it's enough well-known to be clear and immediate. The domain is \emph{.gov} that clearly represent an official government website so its choice is appropriate. Others domains, like \emph{.com}/\emph{.org}/\emph{.it}, doesen't redirect to \thesite{} and seems to be not related to the government. Since the website is a sensible target, i would expect that at least the \emph{.com} redirects to the main one, simply to avoid possible attacks like website cloning.

%----------------------------------------------------------------------------------------
%	HOMEPAGE
%----------------------------------------------------------------------------------------

\section{Homepage}
\label{homepage}

\begin{figure}[p]
\centering 
\centerline{\includegraphics[width=1.4\columnwidth]{homepage-entire-sections}}
\caption[Homepage]{Homepage of \thesite{}}
\label{fig:homepage} 
\end{figure}

The first view of the homepage has a scanning time enough low to shows the 6w's\footnote{included ``How'', \url{http://en.wikipedia.org/wiki/Five_Ws}} in the limit time of 30 seconds.

\begin{itemize}%[noitemsep]
	\item \textbf{Where:} where are we? \\ The subtitle isn't so decriptive but the words \emph{whitehouse} and \emph{President Barack Obama} together clearify where we are. The breadcrumb lack doesn't explain where we are inside the website, this create user disorientation;

	\item \textbf{Who:} who represent the site? \\ In the top-left a subtitle explaining that the representer of the whitehouse is the President Barack Obama. Note that's an image and not plain text, this doesn't respect the common web rules, invalidating text advantages. \\
	The whitehouse logo is in the top-center and it includes a graphical text mentioning, another time, the white house. It correctly links to the homepage but a negative note is about the size: the text is too small and the image size doesn't respect the \emph{target size rule}\footnote{the more important a button is, the bigger it should be.}; i believe that the whitehouse logo is enough important (also as homepage link) to deserve a bigger size;

	\item \textbf{Why:} why should i remain in this website? \\ This is not a commercial website so we are interested in public domain infos about the white house. There're daily updates about relevant polical facts, usually located in the central slideshow and in below sections;

	\item \textbf{What:} what offers the site? \\ This site offers any infos and news that a citizen could request, form the President activity to issues like economy, taxes, defense and political reports;

	\item \textbf{When:} this is about temporal references contained in the site. \\ The ``top news'' section contains frequently updated news and, as mentioned above, the central slideshow offers last facts updates, often supplied with media (e.g., picture, video); 

	\item \textbf{How:} how we can move across the website? \\ The menu is top centered and has a good centered and wide submenus. There's also a search bar at the bottom of the central slideshow.

\end{itemize}

VIDEO 8:30 mins ma i .gov non hanno timer
 |||| TODOOO
bandiera amircana link home
footer logo acquila vicino icone social non cliccabile

Homepage can be divided in the following sections:
	
	\begin{enumerate}
		
		\item \textbf{Header:} logo and website subititle are well located in the ``hot'' top-left area, they correctly links to the homepage. Contacts buttons (i.e. get email updates, contact us) are properly coloured as link and the link area include the icons. \\
		US flags located in the left menu confuse users, indeed it redirects to the homepage while flags are commonly often (wrongly\footnote{\href{http://flagsarenotlanguages.com}{http://flagsarenotlanguages.com}}) with language selection. \\
		Submenus're wide and doesn't unexpectedly disappears when user iteract with them, a bit bothersome are those advertisement-like column, they've imgs so distract users in submenu reading;
		
		\item \textbf{Body:} top body has wide area with some text and media, the button is correctly non-graphical and video lentgh variance is very hight, depends on day, but generally there're videos longher then 3 minutes. This is allowed for a \emph{.gov} site but it has to be said that propose a 32mins\footnote{2014-08-08} video in the top homepage couldn't be a wise choice. Search toolbar is discussed in \ref{searchtools}. \\
		Popular topics section lacks in title link color and the buttons to navigate across topics are too small and require precise mouse positioning, not very comfortable for users. It's correct that imgs contains links and slighlty change colour on the mouse over. Below there's a tab selector, tabs navigation is clear, images are all cliccable but sometimes too small. \\
		Top news section has no coloured links and lacks on the same issue, mentioned above, on news naviation. Curious are the the icons \includegraphics[scale=0.6]{social-rows} in the social area, for a non \emph{twitter} user it's difficult to understand their meaning also because they lead to the same splash screen (discussed in \ref{splashpage}). Another note goes to the photo of the day section icons \includegraphics[scale=0.6]{day-photo-home}, indeed while the plus button enlarge the photo, the info icon isn't a button, it overlays textual infos over the picture and isn't cliccable, so the user is confused and ends to click on it, browser select it and result is unaesthetic;
		
		\item \textbf{Footer:} on the top there're social icons dominated by the seal of the President icon, while socials are cliccable the latter isn't so that's different behaviour of near icons, this confuse users.
		Footer height is oversized due the \emph{issue} column that's higher than the average. The bottom image \emph{www.whitehouse.gov} is just an image witout links to the cited website, furthermore the bottom word's font is too small, 9px, against a lower-bound limit allowed of 10px. 
		
	\end{enumerate}

%----------------------------------------------------------------------------------------
%	INTERNAL PAGES
%----------------------------------------------------------------------------------------

\section{Internal pages}
Most of internal pages are structured in base of the sublevel: 2\textsuperscript{nd} and 3\textsuperscript{rd} levels have their layout that seems to remain unchanged for the most part of the sublevel.
All sublevel except the homepage has a \emph{location breadcrumb} located at the bottom of the menu: it shows your position relative to the homepage. Inline with the breadcrumb there's the search bar, the textbox allows 25 chars before to run into the \emph{guillotine effect}\footnote{when textbox are too small so users are forced to use scrollbars to show up the content.}; furthermore the search tool shows a clear \emph{search} button instead to require a click on the hand lens. 
Every page seems to be reachable from the homepage in most 3 clicks, that's why whe menu is complete and well done. 
Internal pages are reachable from different pages than the home (\emph{deep linking}) so they have to propose again the compulsory informative axis: \emph{who}/\emph{what}/\emph{where}. The website correctly respect those requirement.
Below there're the analysis of three subpages, the first one sublevel is 1 and the second and the third are 2\textsuperscript{nd} level.

	\subsection{/briefing-room}
	\label{primapaginainterna}
	

	\begin{figure}[h!]
	\centering 
	\centerline{\includegraphics[width=2\columnwidth]{1-internal-page-entire}}
	\caption[First internal page: /briefing-room]{First internal page: \emph{/briefing-room}.}
	\label{fig:primapaginainterna} 
	\end{figure}

	The briefing room page (Figure \ref{fig:primapaginainterna}) (\href{http://www.whitehouse.gov/briefing-room}{whitehouse.gov/briefing-room}) provides informations about the President's activity, latest events and public statements. The page has a top-central wide image that immediately recall a President's public speech. A text subtitle explaining what kind of page is, that's good beacause accomplish the \emph{where} compulsory axis meanwhile permit the \emph{common text operations}\footnote{select, copy, paste.}, furthermore a set of social icons are well visible to share contents. \\
	The page content is divided in three columns, that's make user scanning effort more demanding but maybe authors wanted to put on the same footing those contents (media, policy, additional infos). Images are a bit short in height (304x125px) but sufficiently big to show their contents, they are't links so clock over doesn't produce any effects, this is undesirable since imgs appeals clicks. \\
	The text content is well divied in separated, short and titled blocks, that's very good, furthermore every title is a colored link. The overall height and structure is enough compact to ensure a rapid scan. \\
	A three columns bottom-content frame is located in the bottom page, columns are recent video, latest and featured policy. The first one shows a clickable thumbnail and subtitle, the image is 274x153px, clearly too small to be exhaustive. The second one present a list of news with a precise temporal reference. The last one has always a 311x272px img, big enough, clickable image.

	\subsection{/briefing-room/legislation}
	\label{secondapaginainterna}

	\begin{figure}[h!]
	\centering 
	\centerline{\includegraphics[width=2\columnwidth]{2-internal-page-visible}}
	\caption[Second internal page: /briefing-room/legislation]{Second internal page: \emph{/briefing-room/legislation}.}
	\label{fig:secondapaginainterna} 
	\end{figure}

	
	The legislation page (Figure \ref{fig:secondapaginainterna}) (\href{http://www.whitehouse.gov/briefing-room/legislation}{whitehouse.gov/briefing-room/legislation}) is a 3\textsuperscript{rd} sublevel page. This page shows links to three kind of legislation (pending, signed, voted). There's no top images. Layout is three columns type, this isn't a good practice because it confuse the user. On the left side there's a menu for the whole \emph{briefing room} section, it shows every section sublevels. Its height is suited for nowadays content but the voice \emph{Disclosures} contains annuals reports so it's expected to grow every year, this means that the left column is going to constant increase its height; this could produce user dissatisfaction so it's a clear that there's a early design lack. Another bad note is about there aren't coloured links. \\
	The central column shows a good design, there's a clear title, short subtitle and a compact list of titles with a short description. \\
	The right side column is a bit problematic. First, there are three frame titled with an image not text. Second, the \emph{Archives} section is an example of the \emph{lorem ipsum damnation}: probably in design time it was filled with a short example content (e.g., lorem ipsum), next with the passing of the time it grew up a lot; this is an early design defect that'll cause trouble to users, what'll happen within ten years?
	

	\subsection{/issues/technology}
	\label{terzapaginainterna} 


	\begin{figure}[h!]
	\centering 
	\centerline{\includegraphics[width=2\columnwidth]{3-internal-page-visible}}
	\caption[Third internal page: /issues/technology]{Third internal page: \emph{/issues/technology}.}
	\label{fig:terzapaginainterna} 
	\end{figure}
	
	The tecnology page (Figure \ref{fig:terzapaginainterna}) (\href{http://www.whitehouse.gov/issues/technology}{whitehouse.gov/issues/technology}) is a 3\textsuperscript{rd} sublevel page. This page is an article about President's view of tecnology and explain the whitehouse guideline principles. There's no top images. As above, there's a three column layout. At left side there's a menu showing all the first level of \emph{issues} section. This choice is different between the above analyzed section, that's maybe caused to not make it too height. Another time, links aren't coloured, this could confuse users. \\
	The central column has a clear title but a too long subtitle, there's a video with no clear title and it's too long (32 mins). Text content isn't well organized: long text blocks (total 3790 words, at least 15-25 mins to read everything) with small titles, it could be divided in more than one page. Text block is so long that at the bottom page there's a link to jump to top of page. Social sharing icons are locatend at the end of the content so it'll take some time to spot it, this should not happen in the web 2.0 era where content sharing runs very fast. \\
	The right column shows a 272x238px img with a fake button inside, this is big enough but the button contained isn't so intuitive for users. Than there're some related blog posts well presented, with coloured links title and the temporal reference. Next there's the related videos section, video's thumbnails are too small (274x153px) and, in conflict with the previous section video titles are placed below the video thumbnail.
	
	\begin{figure}[h!]
	\centering 
	\centerline{\includegraphics[width=0.8\columnwidth]{3-internal-page-detail}}
	\caption[feedback-form]{feedback-form included in every 3\textsuperscript{rd} sublevel page.}
	\label{fig:terzapaginainternadettaglio} 
	\end{figure}

	A black mark goes to the feedback-form shown in Figure \ref{fig:terzapaginainternadettaglio}, although it's clear how to use it and has a limited number of inputs and texts, its design is a bit strange. The comments text-box cannot be enlarged even though there's some white right space. The way of escape is clear, there's a text button in upper right corner, however users would expect to close this form by clicking outside it, but it'doesent works in that way. Furthermore, the feedback-form behaviour is confusing: clicking the green \emph{give feedback about this page} button (another time, an img button with non-text inside) to show up the form there's a too long delay befefore the form appearence, this happens also when user wants to close the form by clicking the green button.

%----------------------------------------------------------------------------------------
%	GENERALS OBSERVATIONS
%----------------------------------------------------------------------------------------

\newpage
\section{General observations}

	\subsection{Images}
	The majority of images can be clicked but there are pages (e.g., \ref{primapaginainterna}) with no cliccable images. This disorientate users especially in this site where the most part if imgs are cliccable, indeed user have to understand by the context if and why he cannot click over an image.
	Image sizing is generally small, around or below the 210x230px limit, but important images are always big enough to be understood.

	\subsection{Texts}
	Text is visible, clear, big enough, always well contrastwd with the background. There's no sizing buttons and there are too many fonts, sometimes more than 4 (e.g., \ref{terzapaginainterna}).
	There's an abuse of images containing texts and buttons, this as said above, tend to confuse users. 
	There're upper-case sentences, sometimes used as subtitle, but those number is not noteworthy. No keywords highlighted in bold.


	\subsection{Links}
	Links are not always coloured and visited links doesn't change color. This goes against web conventions and increase the user effort to navigate. 

	\subsection{Scroll}
	Pages height is well sized for the high level pages link the home or one sublevel but as user go in deep levels pages grows immoderately. Let's compare the homepage height with a 3\textsuperscript{rd} level page (\ref{terzapaginainterna}). \\
	Entire homepage is shown in Figure \ref{fig:entirehomepage}, entire content is contained in the first two screens, considering that the 3\textsuperscript{rd} screen is due to the \emph{issues} column, this is acceptable.
	A bad mark goes to the \ref{terzapaginainterna} which is 9752px high so, with a 1440x900px monitor, users needs to scroll 11 entire screens, this is unaceptable, it'd better to separate content in diffeent tabs. 

	\begin{figure}[h]
	\centering 
	\centerline{\includegraphics[width=0.9\columnwidth]{homepage-entire-screen}}
	\caption[Screen divided homepage]{Entire Homepage divided by 900px height screen.}
	\label{fig:entirehomepage} 
	\end{figure}

	\subsection{Search tools}
	\label{searchtools}
	Consided the website size, a search tool is present. The site use Bing\footnote{\href{http://en.wikipedia.org/wiki/Bing}{http://en.wikipedia.org/wiki/Bing}} as search engine but the search seems to be personalized for White House requirements so we can consider it as local search, indeed there's no redirects to the search engine page. The only reference to the external search engine at the page bottom there's an uncliccable logo with the subtitle: ``\emph{results by Bing}''. \\
	Search toolbar present in pages above is different from homepage (\ref{homepage}) and others subpages (\ref{primapaginainterna}, \ref{secondapaginainterna}, \ref{terzapaginainterna}). The first one is located below the central wide image and the search button is an hand lens icon (bad); subpages has search bar above the central image, this time with the proper \emph{Search} button. Either search leads to a results page (Figure \ref{fig:search-simple}). In case of an unsuccessful search the page correctly warns the user with a message (Figure \ref{fig:search-no-results}).

	\begin{figure}[h]
	\centering 
	\centerline{\includegraphics[width=2\columnwidth]{search-simple}}
	\caption[Simple search results page]{Simple search results page.}
	\label{fig:search-simple} 
	\end{figure}

	\begin{figure}[h]
	\centering 
	\centerline{\includegraphics[width=2\columnwidth]{search-no-results}}
	\caption[No results search results page]{No results search results page.}
	\label{fig:search-no-results} 
	\end{figure}

	Advanced search (Figure \ref{fig:search-advanced}) is reachable starting from a simple search result page (Figure \ref{fig:search-simple}), this makes advanced search optional and is a positive mark. Notable is the restricted number of selection filters but it isn't explained what exactly means \emph{safe search} in a goverment website. Good thing is the search button instead of the hand lens icon.


	\begin{figure}[h]
	\centering 
	\centerline{\includegraphics[width=2\columnwidth]{search-advanced}}
	\caption[Advanced search page]{Advanced search page}
	\label{fig:search-advanced} 
	\end{figure}


	\subsection{Back button}
	Back button is propely working as expected in every page and never opens the previous page in a detatched window.

	\subsection{Splash page}
	\label{splashpage}
	During the first site visit there aren't any splash screens or annoying registation form, nevertheless a splash screen (Figure \ref{fig:splash-screen}) appears when user click on a link to an exteral web site, in this case a message warns user that is leaving the \thesite{} web server. This message is very annoying and it goes against the web conventions.

	\begin{figure}[h]
	\centering 
	\centerline{\includegraphics[width=1\columnwidth]{splash-screen}}
	\caption[Splash screen]{Splash screen appears when leaving \thesite{}.}
	\label{fig:splash-screen} 
	\end{figure}

	\subsection{Ads}
	This isn't a commercial website, nevertheless the site proposes a lot of content so it's common to have some ads to specific theme (e.g., join military forces, taxes caluclator). Ads are correctly located in the right column and gracefully integrates with the site look. Althought there aren't awful effects like blinking, rotating or on the move texts, it has to be said that sometimes there're cliccable images ads, sometimes there're uncliccable imgs detatched from the link button. This clearly confuse users that usually try to click over images.

	\subsection{404}
	Inserting an unvalid url the webiste correctly shows an error message (Figure \ref{fig:not-found-page}). This scenario could happen when a user follows a broken link, the site properly invite to go to the homepage using the back button or by the proposed link. This 

	\begin{figure}[h]
	\centering 
	\centerline{\includegraphics[width=1.5\columnwidth]{404}}
	\caption[Not found page]{Not found page message.}
	\label{fig:not-found-page} 
	\end{figure}

%----------------------------------------------------------------------------------------
%	SUMMARY
%----------------------------------------------------------------------------------------

\clearpage
\section{Summary}
	\begin{centering}
	{ \huge \bfseries grade: 7.5/10 \\[0.3cm] }
	\end{centering}
	Considering the huge size of the site, the fews defects identified and the complexity and variety of content delived by \thesite{} i consider it as a well designed website. Except the excessive height of some pages, defects detected are not so heavy to spoil navigation.


%----------------------------------------------------------------------------------------
%	LIST OF FIGURES
%----------------------------------------------------------------------------------------

\section{List of figures with file references}
Table \ref{tab:list-of-figures} associate url's with images used in this document:

\begin{table}[ht]
	\caption{List of figures with file references} % title of Table
	\centering % used for centering table
	\begin{centering}
	\begin{tabular}{l c c} % centered columns (4 columns)
		\hline\hline %inserts double horizontal lines
		Figure & File name & URL \\ [0.5ex] % inserts table 
		%heading
		\hline % inserts single horizontal line
			Homepage & homepage-entire-sections.png & \href{http://www.whitehouse.gov/}{link} \\ [1ex]
			First internal page: /briefing-room & 1-internal-page-entire.png & \href{http://www.whitehouse.gov/briefing-room}{link}\\ [1ex]
			Second internal page: /briefing-room/legislation & 2-internal-page-visible.png & \href{http://www.whitehouse.gov/briefing-room/legislation}{link} \\ [1ex]
			Third internal page: /issues/technology & 3-internal-page-visible.png & \href{http://www.whitehouse.gov/issues/technology}{link} \\ [1ex]
			feedback-form & 3-internal-page-detail.png & \href{http://www.whitehouse.gov/issues/technology}{link} \\ [1ex]
			Screen divided homepage & homepage-entire-screen.png & \href{http://www.whitehouse.gov/}{link} \\ [1ex]
			Simple search results page & search-simple.png & \href{http://stackoverflow.com/questions/2640111/url-latex-linebreak-problem}{link} \\ [1ex]
			No results search results page & search-no-results.png & \href{http://search.whitehouse.gov/search?utf8=%E2%9C%93&query=lorem+ipsum&m=&affiliate=wh&commit=Search}{link} \\ [1ex]
			Advanced search page & search-advanced.png & \href{http://search.whitehouse.gov/search/advanced?affiliate=wh&enable_highlighting=true&per_page=20&query=obama}{link} \\ [1ex]
			Splash screen & splash-screen.png & \href{http://www.whitehouse.gov/issues/technology}{link} \\ [1ex]
			Not found page & 404.png & \href{http://www.whitehouse.gov/admin}{link} \\ [1ex]
			 % &  &  \\ [1ex]
			 % &  &  \\ [1ex]
			 % &  &  \\ [1ex]
			 % &  &  \\ [1ex]
			 % &  &  \\ [1ex]
			 % &  &  \\ [1ex]
		\hline %inserts single line
	\end{tabular}
	\end{centering}
	\label{tab:list-of-figures} % is used to refer this table in the text
\end{table}

\end{document}